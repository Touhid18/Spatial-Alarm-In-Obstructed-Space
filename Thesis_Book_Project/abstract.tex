\chapter*{Abstract}
\addcontentsline{toc}{chapter}{Abstract}

\hspace{3ex} Spatial Alarms are an important class of personalized location based services (LBSs) that are triggered by a specific location of a moving user, instead of time. A pedestrian may plan to buy a medicine from a pharmacy while walking through a city or a tourist may want to have the dinner at a restaurant while taking a scenic walk at a unfamiliar place. Spatial alarms enable users to know if a point of interest (POI) such as a pharmacy or a restaurant comes within a specific range of the pedestrian. In this thesis, we introduce an efficient approach to evaluate spatial alarm queries in the obstructed space. Existing work in this area has focused mainly on the Euclidean space and road network models. However, the movement of a pedestrian is restricted with many obstacles like buildings, fences or vehicles. To trigger the spatial alarm for a pedestrian, an LBS needs to continuously check whether the specific type of POI is within the specific range of the current location of the pedestrian, which incurs high query processing overhead. We exploit geometric properties to refine the POI and obstacle search region in the total space and develop a technique to identify the area, where a pedestrian’s movement does not need to retrieve any new POI or obstacle from the LSP. Our approach avoids the retrieval of same POIs and obstacles from the LSP’s databases. We perform extensive experiments using real data sets and show that our approach is significantly faster and requires less IO than a naïve approach.


