\chapter{Conclusion}
\label{chp:conclusion}
In this thesis we have addressed evaluation of spatial alarm in obstructed space for the first time. We have introduced an approach for efficient evaluation of spatial alarm with minimum device wake-ups and duplicate data retrival from the server.Our approach has been tailored for moving clients which is best suited for clients next movement.We have proposed two variations of our approach to incorporate both accuracy and efficient communication bandwidth.Our experimental setup provides a comparative analysis between our approach and the naive approach on different parameters. We have finished our experiment using both real and synthetic datasets.The results of the experiment conducted shows that our proposed approach is better than the naive approach in execution time, IO access and number of server queries by multiple times.  

\vspace{5pt}
\section{Future Directions}
In the future we wish to extend our thesis work in several directions. 

\begin{itemize}
    \item In future, we aim to manipulate the geometrical properties of the regions to find a larger safe region.
    \item We wish to find a better prediction function for approximating the clients' next direction. 
    \item In the future we wish to provide authentication and privacy while accessing the client's location.
    \item We aspire to find an optimization for visibility graph construction for our approach.
    \item In the future we would like to use our approach to solve spatial alarm queries with moving targets. 

\end{itemize}


\endinput
